\chapter{Kesimpulan dan saran}

\section{Kesimpulan}

Berdasarakan hasil dari analisis, implementasi dan pengujian aplikasi IDE UNPAR mobile yang telah dikembangkan, diperoleh kesimpulan sebagai berikut :

\begin{enumerate}
\item Aplikasi IDE UNPAR mobile berhasil dihubungkan dengan situs IDE UNPAR, sehingga pengguna tidak perlu memasukkan tautan untuk IDE UNPAR ketika pertama kali menjalankan aplikasi.

\item Aplikasi IDE UNPAR mobile dikembangkan dengan mengimplementasikan beberapa fitur yang diperoleh dari kuesioner dan yang seperti dibahas pada subbab \ref{feature feedback}. Fitur-fitur tersebut adalah :
	\begin{itemize}
		\item Fitur \textit{PDF scanner}.
		\item  Fitur menu dengan tautan menuju Student Portal UNPAR
	\end{itemize}
	
\item Aplikasi IDE UNPAR telah diluncurkan ke dalam Google Play milik IF UNPAR dengan status \textit{Open testing}.
\end{enumerate}

\section{Saran}

Berdasarkan pengembangan yang telah dilakukan berikut adalah saran-saran untuk pengembangan seterusnya :

\begin{enumerate}
\item Implementasi fitur \textit{PDF scanner} yang memiliki \textit{edge detection}.
\item Melakukan liris pada perangkat bergerak dengan platform iOS.

\end{enumerate}

