\chapter{Implementasi dan Pengujian}

\section{Lingkungan Pengembangan}
Pada subbab ini akan dibahas lingkungan pengembangan yang digunakan dalam penilitian beserta dengan penyesuaian-penyesuainnya.
\subsection{Penyesuaian lingkungan pengembangan}
Spesifikasi lingkungan pengembangan yang digunakan oleh peneliti adalah sebagai berikut :

\begin{enumerate}
	\item \textit{Browser} Chromium.
	\item Git, dengan \textit{fork} yang mengarah kepada \textit{repository} resmi Moodle mobile.
	\item Node.js versi 14.
	\item \textit{Native build tools} Windows.
\end{enumerate} 

Penggunaan versi Node.js yang berstatus LTS sudah didukung pada \textit{branch integration} Moodle mobile. Dukungan untuk versi Node.js berstatus LTS dilakukan oleh Moodle karena Node.js versi 11 sudah tidak berstatus LTS. Perbaruan tersebut belum terintegrasi dalam \textit{branch master} Moodle mobile, karena akan dirilis pada Moodle mobile versi 4. \cite{MoodleTracker:Node11+} 

Moodle mobile dapat dijalankan dengan menggunakan perintah \textit{npm start}.  Perintah \textit{npm start} akan memanggil sebuah \textit{script} yang kemudian menjalankan perintah \textit{npx gulp watch} dan \textit{npx ionic-app-scripts serve -b --devapp --address=0.0.0.0}, namun ketika \textit{npm start} digunakan dalam windows akan tertahan pada \textit{loop} dengan pesan \texttt{[INFO] Waiting for connectivity with npm...	}. Masalah terjebak dalam \textit{loop} tersebut dapat dihindari dengan memanggil perintah \textit{npx gulp watch} dan \textit{npx ionic-app-scripts serve -b --devapp --address=0.0.0.0} secara terpisah. 


\section{Penyesuaian Tema}
Tema bawaan Moodle mobile akan diubah untuk menyesuaiakna seperti apa yang sudah dibahas pada subbab \ref{sec:Kondisi IDE UNPAR}. Implementasi perubahan-perubahan tersebut akan dibahas pada subbab subbab berikut.

\subsection{Penyesuaian skema warna}
Skema warna yang digunakan oleh IDE UNPAR adalah warna putih dengan kode \texttt{\#ffffff}, abu-abu dengan kode \texttt{\#565656} dan hijau dengan kode \texttt{\#0d8722}.  Warna-warna tersebut adalah warna utama yang terlihat ketika membuka IDE UNPAR sebagai mahasiswa. Sehingga warna-warna tersebut akan menjadi fokus utama untuk pengubahan skema warna pada Moodle mobile.

Mengubah warna yang digunakan oleh Moodle mobile dapat dilakukan dengan menambah dan mengganti variable warna yang berada pada \texttt{src/theme/variables.scss}. Karena Moodle mobile sudah menggunakan warna putih dan abu-abu, maka warna yang harus ditambahkan adalah warna hijau. Penambahan warna hijau IDE UNPAR dapat dilakukan dengan menambahkan vairable warna baru dan mengganti nilai variable \texttt{\$core-color}. Perubahan dapat dilihat pada \mbox{Kode \ref{variable-scss}}.

\begin{lstlisting}[frame=single, label ={variable-scss}, caption = Mengubah warna utama menjadi warna hijau ]
diff --git a/src/theme/variables.scss b/src/theme/variables.scss
index 20a2d27e9a..7f3bac0f9d 100644
--- a/src/theme/variables.scss
+++ b/src/theme/variables.scss
@@ -30,6 +30,7 @@ $red:             #cb3d4d;
 $orange:          #f98012; // Accent (never text).
 $yellow:          #fbad1a; // Accent (never text).
 $purple:          #8e24aa; // Accent (never text).
+$green :          #0d8722; // IDE Main color
 
 // Branded apps customization
 // --------------------------------------------------
@@ -52,7 +53,7 @@ $orange-light:    lighten($orange, 10%) !default;
 $yellow-light:    mix($yellow, white, 20%) !default;
 $yellow-dark:     mix($yellow, black, 40%) !default;
 
-$core-color:          $orange !default;
+$core-color:          $green !default;
 $core-color-light:    lighten($core-color, 10%) !default;
 $core-color-dark:     darken($core-color, 10%) !default;
\end{lstlisting}

Mengubah warna melalui file \texttt{variable.scss} tidak juga mengubah warna pada \textit{status bar} di Andorid. Untuk menyesuaikan warna \textit{status bar} Android perlu dilakukan dengan mengubah niali varable \texttt{statusbarbg} pada file \texttt{config.json}. Perubahan dapat dilihat pada \mbox{Kode \ref{status-bar-color}}. 

\begin{lstlisting}[frame=single, label ={status-bar-color}, caption = Variable untuk mengubah warna dari \textit{status bar} pada config.json ]
diff --git a/src/config.json b/src/config.json
index 31402ee681..f2bb898b20 100644
--- a/src/config.json
+++ b/src/config.json
@@ -79,13 +79,13 @@
     "skipssoconfirmation": false,
     "forcedefaultlanguage": false,
     "privacypolicy": "https:\/\/moodle.net\/moodle-app-privacy\/",
-    "notificoncolor": "#f98012",
-    "statusbarbg": false,
-    "statusbarlighttext": false,
-    "statusbarbgios": "#f98012",
-    "statusbarlighttextios": true,
-    "statusbarbgandroid": "#df7310",
-    "statusbarlighttextandroid": true,
+    "notificoncolor": "#0d8722",
+    "statusbarbg": "#0d8722",
+    "statusbarlighttext": "#0d8722",
+    "statusbarbgios": "#0d8722",
+    "statusbarlighttextios": "#0d8722",
+    "statusbarbgandroid": false,
+    "statusbarlighttextandroid": false,
     "statusbarbgremotetheme": "#000000",
     "statusbarlighttextremotetheme": true,
     "enableanalytics": false,
\end{lstlisting}

\subsection{Penyesuaian ikon dan \textit{resource} lainnya}
Selain skema warna dari aplikasi, \textit{resource} seperti ikon dan \textit{spalsh screen} dari aplikasi juga akan diganti. Ikon aplikasi akan menggunakan logo UNPAR dan \textit{splash screen} akan menggunakan logo IDE UNPAR. Langkah pertama yang harus dilakukan adalah menyeidkan file \texttt{icon.png} dan \texttt{spalsh.png} di dalam folder \texttt{resource}. Kemudia untuk menghasilkan ikon dan \textit{splash screen} untuk kedua platform Android dan iOS Cordova menyediakan perintah \texttt{npx ionic cordova resource}. Dengan menjalankan perintah tersebut Cordova akan menghasilkan \textit{resource} yang dibutuhkan. \textit{config.xml} juga akan diubah ketika menjalankan perintah tersebut. Perubahan dapat dilihat pada \mbox{Kode \ref{app-resource}}.

\begin{lstlisting}[frame=single, label ={app-resource}, caption = Menggunakan \textit{resource} yang baru saja dihasilkan di dalam aplikasi ]
diff --git a/config.xml b/config.xml
index 01aeaa0c1a..0c2611b778 100644
--- a/config.xml
+++ b/config.xml
@@ -50,6 +50,7 @@
     </feature>
     <platform name="android">
         <resource-file src="MainActivity.java" target="app/src/main
		/java/com/moodle/moodlemobile/MainActivity.java" />
+        <icon src="resources/icon.png" />
         <resource-file src="google-services.json" 
	target="app/google-services.json" />
         <resource-file 
	src="resources/android/icon/drawable-ldpi-smallicon.png"
	target="app/src/main/res/mipmap-ldpi/smallicon.png" />
         <resource-file 
	src="resources/android/icon/drawable-mdpi-smallicon.png" 
	target="app/src/main/res/mipmap-mdpi/smallicon.png" />
@@ -222,9 +223,22 @@
         <config-file parent="/*" target="AndroidManifest.xml">
             <uses-feature 
		android:name="android.hardware.bluetooth"
		android:required="false" />
         </config-file>
+        <splash density="land-ldpi"
	src="resources
	/android/splash/drawable-land-ldpi-screen.png" />
+        <splash density="land-mdpi"
	src="resources
	/android/splash/drawable-land-mdpi-screen.png" />
+        <splash density="land-hdpi" 
	src="resources
	/android/splash/drawable-land-hdpi-screen.png" />
+        <splash density="land-xhdpi" 
	src="resources
	/android/splash/drawable-land-xhdpi-screen.png" />
+        <splash density="land-xxhdpi" 
	src="resources
	/android/splash/drawable-land-xxhdpi-screen.png" />
+        <splash density="land-xxxhdpi" 
	src="resources
	/android/splash/drawable-land-xxxhdpi-screen.png" />
+        <splash density="port-ldpi"
	src="resources
	/android/splash/drawable-port-ldpi-screen.png" />
+        <splash density="port-mdpi" 
	src="resources
	/android/splash/drawable-port-mdpi-screen.png" />
+        <splash density="port-hdpi" 
	src="resources
	/android/splash/drawable-port-hdpi-screen.png" />
+        <splash density="port-xhdpi" 
	src="resources
	/android/splash/drawable-port-xhdpi-screen.png" />
+        <splash density="port-xxhdpi" 
	src="resources
	/android/splash/drawable-port-xxhdpi-screen.png" />
+        <splash density="port-xxxhdpi" 
	src="resources
	/android/splash/drawable-port-xxxhdpi-screen.png" />
     </platform>
     <platform name="ios">
         <resource-file src="GoogleService-Info.plist" />
+        <icon src="resources/icon.png" />
         <edit-config file="*-Info.plist" 
	mode="merge" target="NSLocationWhenInUseUsageDescription">
             <string>We need your location so you 
	can attach it as part of your submissions.</string>
         </edit-config>
@@ -647,5 +661,52 @@
                 </dict>
             </array>
         </config-file>
+        <icon height="57"
		 src="resources/ios/icon/icon.png" width="57" />
+        <icon height="114" 
		src="resources/ios/icon/icon@2x.png" width="114" />
+        <icon height="20" 
		src="resources/ios/icon/icon-20.png" width="20" />
+        <icon height="40"
		 src="resources/ios/icon/icon-20@2x.png" width="40" />
+        <icon height="60"
		 src="resources/ios/icon/icon-20@3x.png" width="60" />
+        <icon height="29" 
		src="resources/ios/icon/icon-29.png" width="29" />
+        <icon height="58" 
		src="resources/ios/icon/icon-29@2x.png" width="58" />
+        <icon height="87" 
		src="resources/ios/icon/icon-29@3x.png" width="87" />
+        <icon height="48" 
		src="resources/ios/icon/icon-24@2x.png" width="48" />
+        <icon height="55" 
		src="resources/ios/icon/icon-27.5@2x.png" width="55" />
+        <icon height="88" 
		src="resources/ios/icon/icon-44@2x.png" width="88" />
+        <icon height="172" 
		src="resources/ios/icon/icon-86@2x.png" width="172" />
+        <icon height="196" 
		src="resources/ios/icon/icon-98@2x.png" width="196" />
+        <icon height="216" 
		src="resources/ios/icon/icon-108@2x.png" width="216" />
+        <icon height="40"
		 src="resources/ios/icon/icon-40.png" width="40" />
+        <icon height="80" 
		src="resources/ios/icon/icon-40@2x.png" width="80" />
+        <icon height="120" 
		src="resources/ios/icon/icon-40@3x.png" width="120" />
+        <icon height="50" 
		src="resources/ios/icon/icon-50.png" width="50" />
+        <icon height="100"
		 src="resources/ios/icon/icon-50@2x.png" width="100" />
+        <icon height="60" 
		src="resources/ios/icon/icon-60.png" width="60" />
+        <icon height="120" 
		src="resources/ios/icon/icon-60@2x.png" width="120" />
+        <icon height="180" 
		src="resources/ios/icon/icon-60@3x.png" width="180" />
+        <icon height="72" 
		src="resources/ios/icon/icon-72.png" width="72" />
+        <icon height="144" 
		src="resources/ios/icon/icon-72@2x.png" width="144" />
+        <icon height="76" 
		src="resources/ios/icon/icon-76.png" width="76" />
+        <icon height="152" 
		src="resources/ios/icon/icon-76@2x.png" width="152" />
+        <icon height="167" 
		src="resources/ios/icon/icon-83.5@2x.png" width="167" />
+        <icon height="1024" 
		src="resources/ios/icon/icon-1024.png" width="1024" />
+        <splash height="1136" 
	src="resources/ios/splash
	/Default-568h@2x~iphone.png" width="640" />
+        <splash height="1334" 
	src="resources/ios/splash/Default-667h.png" width="750" />
+        <splash height="2688" 
	src="resources/ios/splash
	/Default-2688h~iphone.png" width="1242" />
+        <splash height="1242" 
	src="resources/ios/splash
	/Default-Landscape-2688h~iphone.png" width="2688" />
+        <splash height="1792" 
	src="resources/ios/splash
	/Default-1792h~iphone.png" width="828" />
+        <splash height="828"
	src="resources/ios/splash
	/Default-Landscape-1792h~iphone.png" width="1792" />
+        <splash height="2436" 
	src="resources/ios/splash
	/Default-2436h.png" width="1125" />
+        <splash height="1125" 
	src="resources/ios/splash
	/Default-Landscape-2436h.png" width="2436" />
+        <splash height="2208" 
	src="resources/ios/splash
	/Default-736h.png" width="1242" />
+        <splash height="1242" 
	src="resources/ios/splash
	/Default-Landscape-736h.png" width="2208" />
+        <splash height="1536" 
	src="resources/ios/splash
	/Default-Landscape@2x~ipad.png" width="2048" />
+        <splash height="2048" 
	src="resources/ios/splash
	/Default-Landscape@~ipadpro.png" width="2732" />
+        <splash height="768" 
	src="resources/ios/splash
	/Default-Landscape~ipad.png" width="1024" />
+        <splash height="2048"
	src="resources/ios/splash
	/Default-Portrait@2x~ipad.png" width="1536" />
+        <splash height="2732" 
	src="resources/ios/splash
	/Default-Portrait@~ipadpro.png" width="2048" />
+        <splash height="1024" 
	src="resources/ios/splash
	/Default-Portrait~ipad.png" width="768" />
+        <splash height="960" 
	src="resources/ios/splash
	/Default@2x~iphone.png" width="640" />
+        <splash height="480" 
	src="resources/ios/splash
	/Default~iphone.png" width="320" />
+        <splash height="2732"
	src="resources/ios/splash
	/Default@2x~universal~anyany.png" width="2732" />
     </platform>
 </widget>
\end{lstlisting}

Setelah perubahan tersebut diimplementasikan, dengan melakukan \textit{build} ulang aplikasi menggunakan \texttt{npx ionic cordova build Android} atau \texttt{npx ionic cordova run Android}, maka aplikasi akan menggunakan \textit{resource} yang baru saja dihasilkan.

\section{Pengujian pada Perangkat Bergerak}
Pengujian pada perangkat bergerak akan dilakukan dengan perangkat bergerak denga sistem operasi Android, dikarenakan penulis hanya memiliki akses ke perangkat bergerak dengan sistem operasi Android. Pengujian dapat dilakukan setelah seluruh konfigurasi yang disarankan oleh dokumentasi telah dilakukan dan beberapa konfigurasi tambahan yang tidak disebutkan di dalam dokumentasi Moodle.

Konfigurasi yang tidak tersebut dalam dokumentasi Moodle adalah ketika menajalnkan perintah \texttt{npm run dev:android} akan terjadi \textit{error} dengan pesan \texttt{> No matching client found for package name 'com.ide.mobile'}. Pesan tersebut dimunculkan karena pada file \texttt{config.json} dan \texttt{config.xml} identifikasi aplikasi atau \texttt{app\_id} pada \texttt{config.json} dan \texttt{id} pada \texttt{config.xml} telah diubah menjadi \textbf{com.ide.mobile}. Sehingga ketika \textit{script} dari perintah \texttt{npm run dev:android} dijalankan, \textit{script} tersebut akan melakukan perbandingan antara kedua file tersebut dengan file \texttt{google-service.json}. Karena pada dokumentasi tidak ada perintah untuk mengubah isi dari file \texttt{google-service.json} maka nilai dari variable \texttt{package\_name} tidak akan sesuai dengan identifikasi aplikasi yang sudah diubah. Perubahan dapat dilihat pada \mbox{Kode \ref{google-service.json}}. 

\vspace{1cm}

\begin{lstlisting}[frame=single, label ={google-service.json}, caption = Menyesuaikan \texttt{package\_name} dengan \texttt{app\_id} pada \texttt{google-service.json} ]
      "client_info": {
        "mobilesdk_app_id": "1:111111111111:android:1111111111111111",
        "android_client_info": {
          "package_name": "id.ac.unpar.moodlemobile"
        }
      },
      "oauth_client": 
\end{lstlisting}
