\chapter{Implementasi dan Pengujian}

\section{Lingkungan Pengembangan}
Pada subbab ini akan dibahas lingkungan pengembangan yang digunakan dalam penilitian beserta dengan penyesuaian-penyesuainnya.
\subsection{Penyesuaian Lingkungan Pengembangan}
Spesifikasi lingkungan pengembangan yang digunakan oleh peneliti adalah sebagai berikut :

\begin{enumerate}
	\item \textit{Browser} Chromium.
	\item Git, dengan \textit{fork} yang mengarah kepada \textit{repository} resmi Moodle mobile.
	\item Node.js versi 14.
	\item \textit{Native build tools} Windows.
\end{enumerate} 

Penggunaan versi Node.js yang berstatus LTS sudah didukung pada \textit{branch integration} Moodle mobile. Dukungan untuk versi Node.js berstatus LTS dilakukan oleh Moodle karena Node.js versi 11 sudah tidak berstatus LTS. Perbaruan tersebut belum terintegrasi dalam \textit{branch master} Moodle mobile, karena akan dirilis pada Moodle mobile versi 4. \cite{MoodleTracker:Node11+} 

Menjalankan \texttt{npm run setup} dengan Node.js 14 akan menghasilkan \textit{error} karena file \texttt{package.json} milik Moodle app memiliki aturan seperti pada \mbox{Kode \ref{lst:package-lock:rule}} yang menyatakan bahwa versi node yang dibutuhkan secara ketat adalah Node.js versi 11. Menghapus atau mengubah bagian tersebut akan memungkinkan menjalankan \texttt{npm run setup} dengan versi 14 Node.js. 

\begin{lstlisting}[frame=single, label ={lst:package-lock:rule}, caption = Aturan pada \texttt{package-lock.json} ]
	"engines": {
    	"node": "11.x"
  	}
\end{lstlisting}

Menggunakan Node.js versi 14 menyebabkan \textit{error} terjadi saat menjalankan Moodle app. Pesan \textit{error} yang muncul adalah \texttt{Module '"E:/Gabriel/Development/Skripsi/moodleapp/node \textunderscore modules/cordova-plugin-wkuserscript/types/index"' has no exported member "WKUserS \\ criptInjectionTime"}. \textit{Error} tersebut terjadi karena pada \textit{commit aa77d0f} \texttt{WKUserScriptInjectionTime} telah dihapus\cite{moodleapp:git:wkuserscript}, sehingga menyebabkan Node.js 14 menghentikan proses eksekusi aplikasi karena dianggap adanya \textit{unhandled promise}. Moodle telah mengeluarkan perbaikan untuk masalah ini pada \textit{commit b39c5ef}\cite{moodleapp:git:wkuserscriptfix}. Implementasi fix tersebut dapat dilakukan pada \textit{branch} yang digunakan dengan mengubah kode file \texttt{package-lock.json} menjadi seperti pada \mbox{Kode \ref{lst:package-lock}} dan file \texttt{src/providers/utils/iframe.ts} pada bagian \textit{import} menjadi seperti pada \mbox{Kode \ref{lst:iframe}} dan pada fungsi \texttt{win.WKUserScript.addScript()} menjadi seperti \mbox{Kode \ref{lst:addScript}}.

\begin{lstlisting}[frame=single, label ={lst:package-lock}, caption = Perbaikan pada \texttt{package-lock.json} ]
	"version": "git+https://github.com/moodlemobile/cordova-plugin-wkuserscript.git#1ad47e75a1811cec0a944d3b8b8544b3d5e052ca",
\end{lstlisting}
 
\begin{lstlisting}[frame=single, label={lst:iframe}, caption = Perbaikan pada bagian \textit{import} file \texttt{iframe.ts}]
	import { WKUserScriptWindow } from 'cordova-plugin-wkuserscript';
\end{lstlisting}

\begin{lstlisting}[frame=single, label={lst:addScript}, caption = Perbaikan pada bagian fungsi \texttt{win.WKUserScript.addScript()} file \texttt{iframe.ts}]
	        	win.WKUserScript.addScript({
                    id: 'CoreIframeUtilsRecaptchaScript',
                    file: recaptchaPath,
                    injectionTime: win.WKUserScript.InjectionTime.END,
                });
\end{lstlisting}

Solusi diatas memungkinkan menjalankan Moodle mobile pada \textit{branch master} dengan menggunakan versi Node.js berstatus LTS. Namun karena perbaikan dan pembaruan terbatu diimplementasikan pada \textit{branch Integration}, maka \textit{branch} tersebut yang akan digunakan. 

Menjalankan Moodle mobile dengan Node.js versi 14 pada \textit{branch Integration} akan menyebabkan \textit{error} pada saat aplikasi sedang dalam proses \textit{build}. Masalah yang muncul memiliki pesan \texttt{Node Sass does not yet support your current environment: Windows 64-bit with Unsupported runtime (83)
For more information on which environments are supported please see:
\url{https://github.com/sass/node-sass/releases/tag/v4.13.1}} yang memberitahu kalau Windows 64-bit tidak dapat menjalankan node-sass dengan versi 4.13. Tetapi ketika memeriksa situs npm, ditunjukkan bahwa node-sass versi 4.13 tidak mendukung Node.js versi 14. Versi node-sass yang mendukung Node.js versi 14 adalah node-sass dengan versi 4.14\cite{node:sass}. Untuk mengatasi masalah tersebut cukup mengubah file \texttt{node\_modules/node-sass/package.json} menjadi seperti \mbox{Kode \ref{node:sass version}}.

\begin{lstlisting}[frame=single, label ={node:sass version}, caption = Perubahan versi node-sass pada \texttt{package-lock.json} ]
	{
  "_args": [
    [
      "node-sass@4.14.0",
      "D:\\gabri\\Development\\Skripsi\\moodleapp"
    ]
  ],
  "_development": true,
  "_from": "node-sass@4.14.0",
  "_id": "node-sass@4.14.0",
  "_inBundle": false,
  "_integrity": "sha512-TTWFx+ZhyDx1Biiez2nB0L3YrCZ/8oHagaDalbuBSlqXgUPsdkUSzJsVxeDO9LtPB49+Fh3WQl3slABo6AotNw==",
  "_location": "/node-sass",
  "_phantomChildren": {
    "escape-string-regexp": "1.0.5",
    "has-ansi": "2.0.0",
    "lru-cache": "4.1.5",
    "strip-ansi": "3.0.1",
    "which": "1.3.1"
  },
  "_requested": {
    "type": "version",
    "registry": true,
    "raw": "node-sass@4.14.0",
    "name": "node-sass",
    "escapedName": "node-sass",
    "rawSpec": "4.14.0",
    "saveSpec": null,
    "fetchSpec": "4.14.0"
  },
  "_requiredBy": [
    "/@ionic/app-scripts"
  ],
  "_resolved": "https://registry.npmjs.org/node-sass/-/node-sass-4.14.0.tgz",
  "_spec": "4.14.0",
  "_where": "D:\\gabri\\Development\\Skripsi\\moodleapp",
  "author": {
    "name": "Andrew Nesbitt",
    "email": "andrewnez@gmail.com",
    "url": "http://andrew.github.com"
  },
  "bin": {
    "node-sass": "bin/node-sass"
  },
  "bugs": {
    "url": "https://github.com/sass/node-sass/issues"
  },
  "dependencies": {
    "async-foreach": "^0.1.3",
    "chalk": "^1.1.1",
    "cross-spawn": "^3.0.0",
    "gaze": "^1.0.0",
    "get-stdin": "^4.0.1",
    "glob": "^7.0.3",
    "in-publish": "^2.0.0",
    "lodash": "^4.17.15",
    "meow": "^3.7.0",
    "mkdirp": "^0.5.1",
    "nan": "^2.13.2",
    "node-gyp": "^3.8.0",
    "npmlog": "^4.0.0",
    "request": "^2.88.0",
    "sass-graph": "^2.2.4",
    "stdout-stream": "^1.4.0",
    "true-case-path": "^1.0.2"
  },
  "description": "Wrapper around libsass",
  "devDependencies": {
    "coveralls": "^3.0.2",
    "eslint": "^3.4.0",
    "fs-extra": "^0.30.0",
    "istanbul": "^0.4.2",
    "mocha": "^3.1.2",
    "mocha-lcov-reporter": "^1.2.0",
    "object-merge": "^2.5.1",
    "read-yaml": "^1.0.0",
    "rimraf": "^2.5.2",
    "sass-spec": "git+https://github.com/sass/sass-spec.git#dc2d573",
    "unique-temp-dir": "^1.0.0"
  },
  "engines": {
    "node": ">=0.10.0"
  },
  "files": [
    "bin",
    "binding.gyp",
    "lib",
    "scripts",
    "src",
    "test",
    "vendor"
  ],
  "gypfile": true,
  "homepage": "https://github.com/sass/node-sass",
  "keywords": [
    "css",
    "libsass",
    "preprocessor",
    "sass",
    "scss",
    "style"
  ],
  "libsass": "3.5.4",
  "license": "MIT",
  "main": "lib/index.js",
  "name": "node-sass",
  "nodeSassConfig": {
    "binarySite": "https://github.com/sass/node-sass/releases/download/"
  },
  "repository": {
    "type": "git",
    "url": "git+https://github.com/sass/node-sass.git"
  },
  "scripts": {
    "build": "node scripts/build.js --force",
    "coverage": "node scripts/coverage.js",
    "install": "node scripts/install.js",
    "lint": "eslint bin/node-sass lib scripts test",
    "postinstall": "node scripts/build.js",
    "prepublish": "not-in-install && node scripts/prepublish.js || in-install",
    "test": "mocha test/{*,**/**}.js"
  },
  "version": "4.14.0"
}

\end{lstlisting}

Mengubah versi node-sass akan menyelesaikan \textit{error} yang terjadi dalam proses \textit{build}. Saat aplikasi berjalan akan muncul kembali \textit{error} dengan pesan \texttt{Property 'InjectionTime' does not exist on type 'WKUserScript'}. Setelah ditelursuri, penghapusan \texttt{WKUserScriptInjectionTime} pada \textit{branch integration} tidak pernah terjadi karena variable tersebut berada dalam file \texttt{index.d.ts} yang berada di folder \texttt{node\_ modules}, dan folder terdaftar dalam \texttt{.gitignore}. Mengatasi \textit{error} tersebut dapat dilakukan dengan mengubah file \texttt{index.d.ts} menjadi seperti \mbox{Kode \ref{lindex.d.ts}}. Dengana melakukan perubahan tersebut Moodle mobile akan dapat dijalankan tanpa menyebabkan \textit{error}.

\begin{lstlisting}[frame=single, label ={index.d.ts}, caption = Perubahan pada \texttt{node\_ modules/cordova-plugin-wkuserscript/types/index.d.ts} ]
	
/**
 * Window instance with the plugin object.
 */
export interface WKUserScriptWindow extends Window {
    WKUserScript?: WKUserScript;
}

/**
 * Data to pass for a script.
 */
export interface WKUserScriptData {

    /**
     * An ID to identify the script, to prevent loading the same script twice.
     */
    id: string;

    /**
     * The JS code of the script.
     */
    code?: string;

    /**
     * The path of a JS file to add to the script.
     */
    file?: string;

    /**
     * Injection time. Defaults to WKUserScriptInjectionTime.START.
     */
    injectionTime?: number;
}


/**
 * Provides some functions to add user scripts in WKWebView in iOS.
 */
interface WKUserScript {

    /**
     * Injection times.
     */
    InjectionTime: {
        START: 0;
        END: 1;
    };

    /**
     * Add a user script.
     *
     * @param data Data for the script to add.
     * @return Promise resolved when done.
     */
    addScript(data: WKUserScriptData): Promise<void>;
}

export declare var WKUserScript: WKUserScript;

\end{lstlisting}
