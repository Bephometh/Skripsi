\chapter{Implementasi dan Pengujian}

\section{Lingkungan Pengembangan}
Pada subbab ini akan dibahas lingkungan pengembangan yang digunakan dalam penilitian beserta dengan penyesuaian-penyesuainnya.
\subsection{Penyesuaian Lingkungan Pengembangan}
Spesifikasi lingkungan pengembangan yang digunakan oleh peneliti adalah sebagai berikut :

\begin{enumerate}
	\item \textit{Browser} Chromium.
	\item Git, dengan \textit{fork} yang mengarah kepada \textit{repository} resmi Moodle mobile.
	\item Node.js versi 14.
	\item \textit{Native build tools} Windows.
\end{enumerate} 

Penggunaan versi Node.js yang berstatus LTS sudah didukung pada \textit{branch integration} Moodle mobile. Dukungan untuk versi Node.js berstatus LTS dilakukan oleh Moodle karena Node.js versi 11 sudah tidak berstatus LTS. Perbaruan tersebut belum terintegrasi dalam \textit{branch master} Moodle mobile, karena akan dirilis pada Moodle mobile versi 4. \cite{MoodleTracker:Node11+} 

Moodle mobile dapat dijalankan dengan menggunakan perintah \textit{npm start}.  Perintah \textit{npm start} akan memanggil sebuah \textit{script} yang kemudian menjalankan perintah \textit{npx gulp watch} dan \textit{npx ionic-app-scripts serve -b --devapp --address=0.0.0.0}, namun ketika \textit{npm start} digunakan dalam windows akan tertahan pada \textit{loop} dengan pesan \texttt{[INFO] Waiting for connectivity with npm...	}. Masalah terjebak dalam \textit{loop} tersebut dapat dihindari dengan memanggil perintah \textit{npx gulp watch} dan \textit{npx ionic-app-scripts serve -b --devapp --address=0.0.0.0} secara terpisah. 

Menjalankan Moodle mobile dengan Node.js versi 14 pada \textit{branch Integration} akan menyebabkan \textit{error} pada saat aplikasi sedang dalam proses \textit{build}. Masalah yang muncul memiliki pesan \texttt{Node Sass does not yet support your current environment: Windows 64-bit with Unsupported runtime (83)
For more information on which environments are supported please see:
\url{https://github.com/sass/node-sass/releases/tag/v4.13.1}} yang memberitahu kalau Windows 64-bit tidak dapat menjalankan node-sass dengan versi 4.13. Tetapi ketika memeriksa situs npm, ditunjukkan bahwa node-sass versi 4.13 tidak mendukung Node.js versi 14. Versi node-sass yang mendukung Node.js versi 14 adalah node-sass dengan versi 4.14\cite{node:sass}. Menjalankan ulang \textit{npm run setup} juga tidak menyelesaikan masalah karena ketika node-sass akan diunduh kembali tautan yang digunakan tidak dianggap valid. Untuk mengatasi masalah tersebut cukup mengubah file \texttt{node\_modules/node-sass/package.json} menjadi seperti \mbox{Kode \ref{node:sass version}} dan menjalankan ulang  \textit{npm run setup}. \\ \\

\begin{lstlisting}[frame=single, label ={node:sass version}, caption = Perubahan versi node-sass pada \texttt{package-lock.json} ]
	{
  "_args": [
    [
      "node-sass@4.14.0",
      "D:\\gabri\\Development\\Skripsi\\moodleapp"
    ]
  ],
  "_development": true,
  "_from": "node-sass@4.14.0",
  "_id": "node-sass@4.14.0",
  "_inBundle": false,
  "_integrity": "sha512-TTWFx+ZhyDx1Biiez2nB0L3YrCZ/8oHagaDalbu
BSlqXgUPsdkUSzJsVxeDO9LtPB49+Fh3WQl3slABo6AotNw==",
  "_location": "/node-sass",
  "_phantomChildren": {
    "escape-string-regexp": "1.0.5",
    "has-ansi": "2.0.0",
    "lru-cache": "4.1.5",
    "strip-ansi": "3.0.1",
    "which": "1.3.1"
  },
  "_requested": {
    "type": "version",
    "registry": true,
    "raw": "node-sass@4.14.0",
    "name": "node-sass",
    "escapedName": "node-sass",
    "rawSpec": "4.14.0",
    "saveSpec": null,
    "fetchSpec": "4.14.0"
  },
  "_requiredBy": [
    "/@ionic/app-scripts"
  ],
  "_resolved": "https://registry.npmjs.org/node-sass/
-/node-sass-4.14.0.tgz",
  "_spec": "4.14.0",
  "_where": "D:\\gabri\\Development\\Skripsi\\moodleapp",
  "author": {
    "name": "Andrew Nesbitt",
    "email": "andrewnez@gmail.com",
    "url": "http://andrew.github.com"
  },
  "bin": {
    "node-sass": "bin/node-sass"
  },
  "bugs": {
    "url": "https://github.com/sass/node-sass/issues"
  },
  "dependencies": {
    "async-foreach": "^0.1.3",
    "chalk": "^1.1.1",
    "cross-spawn": "^3.0.0",
    "gaze": "^1.0.0",
    "get-stdin": "^4.0.1",
    "glob": "^7.0.3",
    "in-publish": "^2.0.0",
    "lodash": "^4.17.15",
    "meow": "^3.7.0",
    "mkdirp": "^0.5.1",
    "nan": "^2.13.2",
    "node-gyp": "^3.8.0",
    "npmlog": "^4.0.0",
    "request": "^2.88.0",
    "sass-graph": "^2.2.4",
    "stdout-stream": "^1.4.0",
    "true-case-path": "^1.0.2"
  },
  "description": "Wrapper around libsass",
  "devDependencies": {
    "coveralls": "^3.0.2",
    "eslint": "^3.4.0",
    "fs-extra": "^0.30.0",
    "istanbul": "^0.4.2",
    "mocha": "^3.1.2",
    "mocha-lcov-reporter": "^1.2.0",
    "object-merge": "^2.5.1",
    "read-yaml": "^1.0.0",
    "rimraf": "^2.5.2",
    "sass-spec": "git+https://github.com/sass/sass-spec.git#dc2d573",
    "unique-temp-dir": "^1.0.0"
  },
  "engines": {
    "node": ">=0.10.0"
  },
  "files": [
    "bin",
    "binding.gyp",
    "lib",
    "scripts",
    "src",
    "test",
    "vendor"
  ],
  "gypfile": true,
  "homepage": "https://github.com/sass/node-sass",
  "keywords": [
    "css",
    "libsass",
    "preprocessor",
    "sass",
    "scss",
    "style"
  ],
  "libsass": "3.5.4",
  "license": "MIT",
  "main": "lib/index.js",
  "name": "node-sass",
  "nodeSassConfig": {
    "binarySite": "https://github.com/sass/node-sass/releases/download/"
  },
  "repository": {
    "type": "git",
    "url": "git+https://github.com/sass/node-sass.git"
  },
  "scripts": {
    "build": "node scripts/build.js --force",
    "coverage": "node scripts/coverage.js",
    "install": "node scripts/install.js",
    "lint": "eslint bin/node-sass lib scripts test",
    "postinstall": "node scripts/build.js",
    "prepublish": "not-in-install && node scripts/prepublish.js 
|| in-install",
    "test": "mocha test/{*,**/**}.js"
  },
  "version": "4.14.0"
}

\end{lstlisting}

Mengubah versi node-sass akan menyelesaikan \textit{error} yang terjadi dalam proses \textit{build}. Saat aplikasi berjalan akan muncul kembali \textit{error} dengan pesan \texttt{Property 'InjectionTime' does not exist on type 'WKUserScript'}. Setelah ditelursuri, penghapusan \texttt{WKUserScriptInjectionTime} pada \textit{branch integration} tidak pernah terjadi karena variable tersebut berada dalam file \texttt{index.d.ts} yang berada di folder \texttt{node\_ modules}, dan folder terdaftar dalam \texttt{.gitignore}. Mengatasi \textit{error} tersebut dapat dilakukan dengan mengubah file \texttt{index.d.ts} menjadi seperti \mbox{Kode \ref{index.d.ts}}. Dengana melakukan perubahan tersebut Moodle mobile akan dapat dijalankan tanpa menyebabkan \textit{error}.

\begin{lstlisting}[frame=single, label ={index.d.ts}, caption = Perubahan pada \texttt{node\_ modules/cordova-plugin-wkuserscript/types/index.d.ts} ]
	
/**
 * Window instance with the plugin object.
 */
export interface WKUserScriptWindow extends Window {
    WKUserScript?: WKUserScript;
}

/**
 * Data to pass for a script.
 */
export interface WKUserScriptData {

    /**
     * An ID to identify the script, 
     * to prevent loading the same script twice.
     */
    id: string;

    /**
     * The JS code of the script.
     */
    code?: string;

    /**
     * The path of a JS file to add to the script.
     */
    file?: string;

    /**
     * Injection time. Defaults to WKUserScriptInjectionTime.START.
     */
    injectionTime?: number;
}


/**
 * Provides some functions to add user scripts in WKWebView in iOS.
 */
interface WKUserScript {

    /**
     * Injection times.
     */
    InjectionTime: {
        START: 0;
        END: 1;
    };

    /**
     * Add a user script.
     *
     * @param data Data for the script to add.
     * @return Promise resolved when done.
     */
    addScript(data: WKUserScriptData): Promise<void>;
}

export declare var WKUserScript: WKUserScript;

\end{lstlisting}

\section{Penyesuaian Tema}
Tema bawaan Moodle mobile akan diubah untuk menyesuaiakna seperti apa yang sudah dibahas pada subbab \ref{sec:Kondisi IDE UNPAR}. Implementasi perubahan-perubahan tersebut akan dibahas pada subbab subbab berikut.

\subsection{Penyesuaian skema warna}
Skema warna yang digunakan oleh IDE UNPAR adalah warna putih dengan kode \texttt{\#ffffff}, abu-abu dengan kode \texttt{\#565656} dan hijau dengan kode \texttt{\#0d8722}.  Warna-warna tersebut adalah warna utama yang terlihat ketika membuka IDE UNPAR sebagai mahasiswa. Sehingga warna-warna tersebut akan menjadi fokus utama untuk pengubahan skema warna pada Moodle mobile.

Mengubah warna yang digunakan oleh Moodle mobile dapat dilakukan dengan menambah dan mengganti variable warna yang berada pada \texttt{src/theme/variables.scss}. Karena Moodle mobile sudah menggunakan warna putih dan abu-abu, maka warna yang harus ditambahkan adalah warna hijau. Penambahan warna hijau IDE UNPAR dapat dilakukan dengan menambahkan vairable warna baru dan mengganti nilai variable \texttt{\$core-color} seperti pada \mbox{Kode \ref{variablse.scss}}.

\begin{lstlisting}[frame=single, label ={variable.scss}, caption = Mengubah warna utama menjadi warna hijau ]
$green :          #0d8722; // IDE Main color

... 

$core-color:          $green !default;
\end{lstlisting}

Mengubah warna melalui file \texttt{variable.scss} tidak juga mengubah warna pada \textit{status bar} di Andorid. Untuk menyesuaikan warna \textit{status bar} Android perlu dilakukan dengan mengubah niali varable \texttt{statusbarbg} pada file \texttt{config.json} seperti pada \mbox{Kode \ref{status-bar-color}}. 

\begin{lstlisting}[frame=single, label ={status-bar-color}, caption = Variable untuk mengubah warna dari \textit{status bar} pada config.json ]
    "notificoncolor": "#0d8722",
    "statusbarbg": "#0d8722",
    "statusbarlighttext": "#0d8722",
    "statusbarbgios": "#0d8722",
    "statusbarlighttextios": "#0d8722",
    "statusbarbgandroid": false,
    "statusbarlighttextandroid": false,
\end{lstlisting}

\section{Pengujian pada Perangkat Bergerak}
Pengujian pada perangkat bergerak akan dilakukan dengan perangkat bergerak denga sistem operasi Android, dikarenakan penulis hanya memiliki akses ke perangkat bergerak dengan sistem operasi Android. Pengujian dapat dilakukan setelah seluruh konfigurasi yang disarankan oleh dokumentasi telah dilakukan dan beberapa konfigurasi tambahan yang tidak disebutkan di dalam dokumentasi Moodle.

Konfigurasi yang tidak tersebut dalam dokumentasi Moodle adalah ketika menajalnkan perintah \texttt{npm run dev:android} akan terjadi \textit{error} dengan pesan \texttt{> No matching client found for package name 'com.ide.mobile'}. Pesan tersebut dimunculkan karena pada file \texttt{config.json} dan \texttt{config.xml} identifikasi aplikasi atau \texttt{app\_id} pada \texttt{config.json} dan \texttt{id} pada \texttt{config.xml} telah diubah menjadi \textbf{com.ide.mobile}. Sehingga ketika \textit{script} dari perintah \texttt{npm run dev:android} dijalankan, \textit{script} tersebut akan melakukan perbandingan antara kedua file tersebut dengan file \texttt{google-service.json}. Karena pada dokumentasi tidak ada perintah untuk mengubah isi dari file \texttt{google-service.json} maka nilai dari variable \texttt{package\_name} tidak akan sesuai dengan identifikasi aplikasi yang sudah diubah. Perubahan dapat dilihat pada \mbox{Kode \ref{google-service.json}}. 

\vspace{1cm}

\begin{lstlisting}[frame=single, label ={google-service.json}, caption = Menyesuaikan \texttt{package\_name} dengan \texttt{app\_id} pada \texttt{google-service.json} ]
      "client_info": {
        "mobilesdk_app_id": "1:111111111111:android:1111111111111111",
        "android_client_info": {
          "package_name": "id.ac.unpar.moodlemobile"
        }
      },
      "oauth_client": 
\end{lstlisting}