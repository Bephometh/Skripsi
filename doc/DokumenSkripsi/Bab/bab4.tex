\chapter{Perancangan}
\label{perancangan} 

\section{Struktur Proyek Moodle Mobile}
\label{struktur proyek}

Proyek Moodle Mobile terdiri dari kumpulan file-file dan direktori-direktori utama yang mengandung fungsi dan \textit{source code} untuk aplikasi Moodle Mobile, platform Android dan iOS, alat untuk mebangun perangkat lunak, dan konfigurasi. Struktur dapat dilihat di bawah.


\begin{center}
\begin{minipage}{.6\textwidth}
config/                      \\                                         
desktop/                      \\                                        
gulp/                                                                 \\
hooks/                                     \\                           
node\_modules/                      \\                                   
platforms/                               \\                             
plugins/                                     \\                         
resources/                               \\                             
scripts/                                       \\                       
src/                                                 \\                 
www/                                             \\                    
.dockerignore                                      \\                  
.gitattributes                                  \\                     
.gitignore                                           \\                
.npmrc                                               \\                
.travis.yml                                          \\                
config.xml                                               \\  
Dockerfile                                               \\
google-services.json                            \\                     
GoogleService-Info.plist                        \\                     
gulpfile.js                                            \\              
ionic.config.json                                       \\             
LICENSE                                                  \\            
licenses.json                                              \\          
MainActivity.java                                       \\             
NOTICE                                                         \\      
package-lock.json                                     \\               
package.json                                            \\             
PACKAGE\_PROBLEMS.md                        \\                          
README.md                                        \\                                                                     
tsconfig.json                                           \\             
tslint.json                                             \\             
upgrade.txt                                     \\
\end{minipage}                                                          
\end{center}
Perubahan pada proyek akan sering dilakukan pada direktori \texttt{src}, dan \texttt{config.xml}. Direktori \texttt{src} berisi kode-kode utama dari aplikasi Moodle mobile dan konfigurasinya. Konfigurasi dari Moodle mobile sendiri diatur oleh file \texttt{config.json} di dalam folder \texttt{src}. File \texttt{Config.xml} berfungsi untuk mengatur konfigurasi dari aplikasi Cordova. Dikarenakan Moodle mobile menggunakan Cordova maka pengaturan saat melakukan \textit{build} untuk perangkat bergerak akan diambil dari file \texttt{config.xml}. 

Perubahan juga akan terjadi pada direktori \texttt{resources} dan file \texttt{ionic.config.json}. Direktori \texttt{resource} menyimpan sumber untuk ikon dan \textit{splashscreen} yang akan digunakan oleh aplikasi ketika di-\textit{build} untuk platform Android maupun iOS. File \texttt{ionic.config.json} adalah file konfigurasi yang digunakan oleh  Ionic CLI (\textit{Comman Line Interface}), dimana nama aplikasi dan identifikasi aplikasi akan disimpan disana dan dirujuk oleh Ionic CLI ketika digunakan.

Direktori \texttt{node\_modules} juga akan mengalami perubahan namun jarang dilakukan perubahan secara langsung. Dikarenakan direktori tersebut menyimpan \textit{packages} yang dikelola oleh npm. Perubahan yang akan terjadi pada folder \texttt{node\_modules} adalah perubahan seperti yang dibahas di subbab \ref{moodle docs:env}.

\subsection{Perubahan pada \texttt{config.xml}}
Perubahan pada file \texttt{config.xml} akan berpengaruh ketika Cordova membangun aplikasi untuk Android dan iOS karena file \texttt{config.xml} mengatur aturan dan \textit{resource} apa saja yang akan digunakan oleh aplikasi di kedua platform tersebut. 

Perubahan yang dilakukan di dalam file ini adalah sebagai berikut : 

\begin{itemize}
\item Versi aplikasi untuk Android dan iOS akan dimulai dengan versi 1.0.0
\item Nama dari aplikasi adalah \textbf{IDE UNPAR Mobile}
\item Deskripsi aplikasi adalah \textbf{IDE UNPAR app}
\item Penulis aplikasi adalah \textbf{Gabriel Panji Lazuardi} dengan email \textbf{73160068@student.unpar.ac.id}
\item Ikon yang akan digunakan oleh aplikasi dalam platform Android dan iOS.
\item \textit{Splashscreen} yang akan digunakan aplikasi dalam platform Android dan iOS.
\end{itemize}  

\subsection{Perubahan pada direktori \texttt{src}}

Karena direktori ini mengandung \textit{source code} utama dari aplikasi, seluruh perubahan yang bersifat menambah, mengubah ataupun mengahpus fungsi di dalam aplikasi akan terjadi di dalam direktori \texttt{src}. Seluruh fungsi utama aplikasi terletak pada direktori \texttt{src/core}. Untuk pengaturan tema dari aplikasi terletak pada direktori \texttt{src/theme}. Dan folder \texttt{src/app} mengandung \textit{source code} yang akan dijalankan secara \textit{native} pada platform Android dan iOS seperti mengatur warna dari \textit{status bar}.	

Dalam folder \texttt{src/core} setiap fungsi dari Moodle mobile juga dipisah ke dalam direktori masing-masing. Setiap direktori dari fungsi-fungsi tersebut akan memiliki \textit{handler} dan \textit{helper}. \textit{Handler} dari sebuah fungsi bersifat seperti \textit{controller} dari pola desain MVC (\textit{Model-view-controller} perangkat lunak. Sehingga perubahan yang akan dilakukan pada file \textit{handler} adalah perubahan yang bersifat menerima dan memberi data dari dan untuk \textit{view}. Sedangkan \textit{helper} bersifat seperti \textit{model} pada pola desain MVS, sehingga perubahan-perubahan pada \textit{helper} Moodle mobile akan bersifat untuk mengolah dan mengembalikan data.

\section{Penerapan fitur-fitur tambahan dari umpan balik}

Berdasarkan yang dibahas pada subbab \ref{feature feedback} beberapa saran fitur akan diimplementasikan pada aplikasi IDE UNPAR. Fitur-fitur tersebut adalah PDF \textit{scanner} dan .

\subsection{PDF \textit{scanner}}
Implementasi fitur ini akan dilakukan dengan bantuan \textit{plugin} Cordova \texttt{cordova-pdf-scanner}. \textit{Plugin} ini dipilih karena memiliki pengaturan paling sederhana. \textit{Plugin} bekerja dengan mengubah HTML mentah ataupun URL menjadi PDF. \textit{Plugin} ini juga dapat mengembalikan hasil pengubahan URL atau HTML mentah ke dalam bentuk base64 agar dapat diunggah kedalam server IDE UNPAR. 

Implementasi fitur PDF \textit{scanner} ini akan dilakukan di direktori \texttt{src/core/fileuploader/}. Dimana \textit{handler} untuk fitur akan disimpan di \texttt{src/core/fileuploader/providers/scanner-handler.ts}. Untuk fungsi memindai gambar untuk diubah menjadi PDF sendiri akan disimpan di \\ \texttt{src/core/fileuploader/providers/helper.ts}. Dengan mengimplementasikan fitur ini pada direktori tersebut, maka fitur ini akan dapat digunakan untuk mengunggah file PDF ke dalam file pribadi pengguna di dalam server IDE UNPAR atau mengungghanya langsung sebagai bentuk submisi tugas atau quiz.\textit{Flow chart} untuk fitur dapat dilihat pada Gambar \ref{fig:scan:flowchart}. 

\begin{figure}[H] 
	\centering  
	\includegraphics[scale=0.6]{FlowChart-ScanPDF.png}  
	\caption[FlowChart untuk fitur PDF \textit{scanner}] {FlowChart untuk fitur PDF \textit{scanner}} 
	\label{fig:scan:flowchart} 
\end{figure} 

File \texttt{src/core/fileuploader/providers/scanner-handler.ts} dapat dilihat pada \mbox{Kode \ref{scanner-handler}}. 

\begin{lstlisting}[language=diff, frame=single, label ={scanner-handler}, caption = File \texttt{scanner-handler.ts} ]

diff --git a/src/core/fileuploader/providers/scanner-handler.ts 
b/src/core/fileuploader/providers/scanner-handler.ts
new file mode 100644
index 0000000000..6cfcbdbc08
--- /dev/null
+++ b/src/core/fileuploader/providers/scanner-handler.ts
@@ -0,0 +1,68 @@
+// (C) Copyright 2015 Moodle Pty Ltd.
+//
+// Licensed under the Apache License, Version 2.0 (the "License");
+// you may not use this file except in compliance with the License.
+// You may obtain a copy of the License at
+//
+//     http://www.apache.org/licenses/LICENSE-2.0
+//
+// Unless required by applicable law or agreed to in writing, software
+// distributed under the License is distributed on an "AS IS" BASIS,
+// WITHOUT WARRANTIES OR CONDITIONS OF ANY KIND, either express or implied.
+// See the License for the specific language governing permissions and
+// limitations under the License.
+
+import { Injectable } from '@angular/core';
+import { CoreAppProvider } from '@providers/app';
+import { CoreUtilsProvider } from '@providers/utils/utils';
+import { CoreFileUploaderHandler, CoreFileUploaderHandlerData } 
  from './delegate';
+import { CoreFileUploaderHelperProvider } from './helper';
+
+@Injectable()
+export class CoreFileUploaderScannerHandler implements 
  CoreFileUploaderHandler {
+    name = 'CoreFileUploaderScanner';
+    priority = 1800;
+
+    constructor(private appProvider: CoreAppProvider, 
+    private utils: CoreUtilsProvider,
+     private uploaderHelper: CoreFileUploaderHelperProvider) { }
+
+    /**
+     * Whether or not the handler is enabled on a site level.
+     *
+     * @return True or promise resolved with true if enabled.
+     */
+    isEnabled(): boolean | Promise<boolean> {
+        return this.appProvider.isMobile() || 
+	this.appProvider.canGetUserMedia();
+    }
+
+    /**
+     * Given a list of mimetypes, return the ones
+     *  that are supported by the handler.
+     * @param mimetypes List of mimetypes.
+     * @return Supported mimetypes.
+     */
+    getSupportedMimetypes(mimetypes: string[]): string[] {
+        return mimetypes;
+    }
+
+    /**
+     * Get the data to display the handler.
+     *
+     * @return Data.
+     */
+    getData(): CoreFileUploaderHandlerData {
+        return {
+            title: 'core.fileuploader.scanner',
+            class: 'core-fileuploader-scanner-handler',
+            icon: 'qr-scanner',
+            action: (maxSize?: number, upload?: 
+	    boolean, allowOffline?: boolean, 
+            mimetypes?: string[]): Promise<any> => {
+                return this.uploaderHelper.scanImage
+	        (maxSize, upload, mimetypes).then((result) => {
+                    return {
+                        treated: true,
+                        result: result
+                    };
+                });
+            }
+        };
+    }
+}
\end{lstlisting} 

Perubahan pada \texttt{src/core/fileuploader/providers/helper.ts} dapat dilihat pada \mbox{Kode \ref{fileuploader-helper}}. 

\begin{lstlisting}[language=diff, frame=single, label ={fileuploader-helper}, caption = Perubahan pada \texttt{src/core/fileuploader/providers/helper.ts} ]
diff --git a/src/core/fileuploader/providers/helper.ts 
b/src/core/fileuploader/providers/helper.ts
index fe28bcb3e4..e69ee619aa 100644
--- a/src/core/fileuploader/providers/helper.ts
+++ b/src/core/fileuploader/providers/helper.ts
+    /**
+     * 
+     * @param maxSize Max size of the upload. -1 for no max size.
+     * @param upload True if file should be uploaded,
+     *  false to return to picked file.
+     * @param mimetypes List of supported mimetypes.  
+     * @return Promise solved when done.
+     */
+    scanImage(maxSize : number,
+    upload?:boolean, mimetypes?: string[]): Promise<any>{
+
+        const camOpts: CameraOptions = {
+            quality: 50,
+            destinationType: this.camera.DestinationType.FILE_URI,
+            correctOrientation: true
+        };
+
+            // Determine the mediaType based on the mimetypes.
+            if (imageSupported && !videoSupported) {
+                camOpts.mediaType = this.camera.MediaType.PICTURE;
+            } else if (!imageSupported && videoSupported) {
+                camOpts.mediaType = this.camera.MediaType.VIDEO;
+            } else if (CoreApp.instance.isIOS()) {
+                // Only get all media in iOS 
+	       //because in Android using this option 
+	      //allows uploading any kind of file.
+                camOpts.mediaType = this.camera.MediaType.ALLMEDIA;
+            }
+        }
+
+        return this.fileUploaderProvider
+	.getPicture(camOpts).then((path) => {
+            const error = this.fileUploaderProvider
+	    .isInvalidMimetype(mimetypes, path); 
+	     // Verify that the mimetype is supported.
+            if (error) {
+                return Promise.reject(error);
+            }
+                const html = '<html> <img src="'+ path +
+	        '" style="width: 100%; height 100%"> </html>';
+                this.logger.debug(html);
+                var currentDate = new Date();
+                return cordova.plugins.pdf.fromData(html,{
+                    fileName : 'my-pdf'+currentDate.getTime()+'.pdf',
+                    landscape : "portrait",
+                    type : "base64" 
+		//Using this type because the document 
+		//will be uploaded right away.
+                }).then((base64)=>{   
+                    //converting to blob
+                    const blob = this.base64ToBlob(base64);
+                    
+                    const contentType = 'application/pdf'
+                    const folderPath = "Download/my-pdf"+currentDate
+		.getTime()+".pdf";
+                    return this.fileProvider.writeFile(folderPath, blob)
+		.then((fileEntry)=>{
+                        const options = this.fileUploaderProvider
+			.getFileUploadOptions(fileEntry.nativeURL, +	
+			'mypdf'+currentDate.getTime()+'.pdf',  
+			contentType, true);
+                        if(upload){;
+                            this.logger.debug("uploaded");
+                            return this.uploadFile
+			(fileEntry.nativeURL, -1, false, options);
+                        } else {
+                            // Copy or move the file 
+			to our temporary folder.
+                            this.logger.debug("Copy to temp");
+                            return this.copyToTmpFolder
+			('Download/', false, maxSize, 'pdf', options);
+                        }
+                    
+                    },
+                    (error) => {
+                        this.logger.error(error);
+                    });
+                },(error) => {
+                    const defaultError = 'core.fileuploader
+		.errorcapturingimage';
+                    console.error(error);
+                    return this.treatImageError(error, defaultError);
+                });
+        });
+    
+    }
+
\end{lstlisting}

Setelah melakukan perubahan-perubahan diatas. Fungsi PDF \textit{scanner} akan muncul pada menu aplikasi IDE UNPAR. Namun nama fungsi akan muncul sebagai \texttt{core.fileuploader.scanner} seperti yang terlihat di fungsi \texttt{getData()} pada \mbox{Kode \ref{scanner-handler}}. Untuk mengubah nama fungsi pada menu dierplukan perubahan pada \texttt{src/core/fileuploader/lang/en.json} dan pada \texttt{scripts/langindex.json}. Perubahan-perubahan pada file tersebut dilakukan dengan tujuna agar Moodle mobile dapat mengenali \textit{title} \texttt{core.fileuploader.scanner} dan menterjemahkan ke dalam bahasa yang sesuai.  Perubahan pada \texttt{scripts/langindex.json} dapat dilihat pada \mbox{Kode \ref{langindex.json}} dan perubahan pada \texttt{src/core/fileuploader/lang/en.json} dapat dilihat pada\mbox{Kode \ref{fileuploader-lang-eng}}.

\begin{lstlisting}[language=diff, frame=single, label ={langindex.json}, caption = Perubahan pada file \texttt{langindex.json} ]
diff --git a/scripts/langindex.json b/scripts/langindex.json
index 0a3a21d16d2..2abd385d7a6 100644
--- a/scripts/langindex.json
+++ b/scripts/langindex.json
@@ -1600,6 +1600,7 @@
   "core.fileuploader.uploading": "local_moodlemobileapp",
   "core.fileuploader.uploadingperc": "local_moodlemobileapp",
   "core.fileuploader.video": "local_moodlemobileapp",
+  "core.fileuploader.scanner" : "local_moodlemobileapp",
   "core.filter": "moodle",
   "core.folder": "moodle",
   "core.forcepasswordchangenotice": "moodle",
\end{lstlisting} 

\begin{lstlisting}[language=diff, frame=single, label ={fileuploader-lang-eng}, caption = Perubahan pada file \texttt{src/core/fileuploader/lang/en.json} ]
diff --git a/src/core/fileuploader/lang/en.json
 b/src/core/fileuploader/lang/en.json
index 22d14df4a11..f0b31d17abf 100644
--- a/src/core/fileuploader/lang/en.json
+++ b/src/core/fileuploader/lang/en.json
@@ -25,5 +25,6 @@
     "uploadafile": "Upload a file",
     "uploading": "Uploading",
     "uploadingperc": "Uploading: {{$a}}%",
-    "video": "Video"
+    "video": "Video",
+    "scanner" : "Scan PDF"
 }
\end{lstlisting} 

\subsection{Link menuju Student Portal UNPAR}
Seperti yang dibahas pada \ref{absesnsi IDE} akan diimplementasikan sebuah tautan pada menu aplikasi IDE UNPAR yang akan mengarahkan pengguna ke \url{https://studentportal.unpar.ac.id}. Selain menambahkan tautan, pilihan \texttt{change site} pada menu aplikasi akan dihapus karena dirasa pengguna tidak membutuhkan pilihan tersebut.

Perubahan yang harus dilakukan adalah menambahkan sebuah tag HTML \texttt{<a>} yang akan mengarahkan pengguna ke \url{https://sutdentportal.unpar.ac.id} pada \texttt{src/core/mainmenu\\/pages/more/more.html}. Dalam tag HTML \texttt{<a>} tersebut ditambahkan atribut \texttt{(click)} dengan nilai fungsi yang akan dipanggil dari \textit{component} Angular ketika elemen tersebut ditekan. Perubahan pada file \texttt{src/core/mainmenu\\/pages/more/more.html} dapat dilihat pada \mbox{Kode \ref{more-view}}.

\begin{lstlisting}[language=diff, frame=single, label ={more-view}, caption = Perubahan pada file \texttt{src/core/mainmenu/pages/more/more.html} ]
diff --git a/src/core/mainmenu/pages/more/more.html
 b/src/core/mainmenu/pages/more/more.html
index 4b2949e7a04..30c80e2e5d0 100644
--- a/src/core/mainmenu/pages/more/more.html
+++ b/src/core/mainmenu/pages/more/more.html
@@ -39,6 +39,10 @@ <h2>{{ 'core.scanqr' | translate }}</h2>
             <ion-icon name="globe" item-start aria-hidden="true">
	</ion-icon>
             <h2>{{ 'core.mainmenu.website' | translate }}</h2>
         </a>
+        <a *ngIf="showWeb" ion-item (click)="openStudentPortal()" 
+ 	core-link autoLogin="yes" 
+ 	title="{{ 'core.mainmenu.studentportal' | translate }}">
+            <ion-icon name="md-school" item-start aria-hidden="true">
+ 	</ion-icon>
+            <h2>{{ 'core.mainmenu.studentportal' | translate }}</h2>
+        </a>
         <a *ngIf="showHelp" ion-item [href]="docsUrl" 
	core-link autoLogin="no" 
	title="{{ 'core.mainmenu.help' | translate }}">
             <ion-icon name="help-buoy" item-start aria-hidden="true">
	</ion-icon>
             <h2>{{ 'core.mainmenu.help' | translate }}</h2>
@@ -46,10 +50,6 @@ <h2>{{ 'core.mainmenu.help' | translate }}</h2>
         <a ion-item (click)="openSitePreferences()" 
	title="{{ 'core.settings.preferences' | translate }}">
             <core-icon name="fa-wrench" item-start></core-icon>
             <h2>{{ 'core.settings.preferences' | translate }}</h2>
-        </a>
-            <a ion-item (click)="logout()" 
- 	   title="{{ logoutLabel | translate }}">
-            <ion-icon name="log-out" item-start aria-hidden="true">
- 	 </ion-icon>
-            <h2>{{ logoutLabel | translate }}</h2>
         </a>
         <ion-item-divider></ion-item-divider>
         <a ion-item (click)="openAppSettings()" 
	title="{{ 'core.settings.appsettings' | translate }}">
 }
\end{lstlisting} 

Setelah melakukan perubahan pada \texttt{src/core/mainmenu/pages/more/more.html}. Langkah selanjutnya adalah menambahkan fungsi \texttt{openStudentPortal()} pada \texttt{src/core/mainmenu/pages/more/more.ts}. Perubahan dapat dilihat di dalam \mbox{Kode \ref{more-component}}.

\begin{lstlisting}[language=diff, frame=single, label ={more-component}, caption = Perubahan pada file \texttt{src/core/mainmenu/pages/more/more.ts} ]
diff --git a/src/core/mainmenu/pages/more/more.ts
b/src/core/mainmenu/pages/more/more.ts
index a3b786c0717..aa700ff2f08 100644
--- a/src/core/mainmenu/pages/more/more.ts
+++ b/src/core/mainmenu/pages/more/more.ts
@@ -207,4 +207,11 @@ export class CoreMainMenuMorePage implements 
OnDestroy {
     logout(): void {
         this.sitesProvider.logout();
     }
+
+    /**
+     * Open https://studentportal.unpar.ac.id
+     */
+    openStudentPortal() : void{
+        this.utils.openInBrowser( "https://studentportal.unpar.ac.id" );
+    }
 }
\end{lstlisting} 

Menu dengan tautan menuju \url{https://studentportal.unpar.ac.id} sekarang akan muncul, namun dengan nama \texttt{core.mainmenu.studentportal} karena aplikasi tidak mengetahui harus diterjemahkan sebagai kalimat apa. Untuk mengatasi masalah tersebut perlu ditambahkannya \texttt{core.mainmenu.studentportal} di dalam \texttt{src/core/mainmenu/lang/en.json} seperti pada \mbox{Kode \ref{stuper-index}}. Dan menambahkannya juga ke dalam \texttt{scripts/langindex.json} yang ditunjukkan pada  \mbox{Kode \ref{stupor-index-script}}

\begin{lstlisting}[language=diff, frame=single, label ={stupor-index}, caption = Menambahkan \texttt{core.mainmenu.studentportal} pada file  \texttt{src/core/mainmenu/lang/en.json} ]
diff --git a/src/core/mainmenu/lang/en.json 
b/src/core/mainmenu/lang/en.json
index 4ff96fbf7b4..cfbdfb5ac3a 100644
--- a/src/core/mainmenu/lang/en.json
+++ b/src/core/mainmenu/lang/en.json
@@ -2,5 +2,6 @@
     "changesite": "Change site",
     "help": "Help",
     "logout": "Log out",
-    "website": "Website"
+    "website": "Website", 
+    "studentportal": "Student portal UNPAR"
 }
\end{lstlisting} 

\begin{lstlisting}[language=diff, frame=single, label ={stupor-index-script}, caption = Menambahkan \texttt{core.mainmenu.studentportal} pada file  \texttt{scripts/langindex.json} ]
diff --git a/scripts/langindex.json b/scripts/langindex.json
index 2abd385d7a6..e155ef93a01 100644
--- a/scripts/langindex.json
+++ b/scripts/langindex.json
@@ -1862,6 +1862,7 @@
   "core.mainmenu.help": "moodle",
   "core.mainmenu.logout": "moodle",
   "core.mainmenu.website": "local_moodlemobileapp",
+  "core.mainmenu.studentportal" : "local_moodlemobileapp",
   "core.maxfilesize": "moodle",
   "core.maxsizeandattachments": "moodle",
   "core.min": "moodle",
\end{lstlisting} 
