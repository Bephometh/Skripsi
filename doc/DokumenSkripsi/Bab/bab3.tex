\chapter{Analisis}
\label{chap:analsia}

\section{Lingkungan Pengembangan}
Kustomisasi Moodle mobile dimulai dengan menyiapkan lingkungan pengembangan agar pengembang dapat mengubah \textit{source code}, melakukan pengujian menggunakan \textit{browser} pada mesin, dan \textit{deploy} aplikasi ke pergangkat seluler. Kebutuhan yang diperlukan untuk lingkungan pengembangan Moodle mobile adalah sebagai berikut :

\begin{enumerate}
	\item Adanya \textit{browser} untuk pengembangan.
	\item Git untuk \textit{source control} aplikasi.
	\item Node.js untuk menjalankan \textit{script} JavaScript tanpa \textit{browser}.
	\item Mesin yang menggunakan sistem operasi Windows akan membutuhkan alat \textit{native build}.
	\item Cocoapods pada Mac untuk mengelola \textit{dependency}.
	\item libsecret untuk sistem operasi linux agar dapat melakukan \textit{push} URL \textit{diff} atau perbedaan antar file dalam \textit{repository} lokal dan \textit{remote}.
\end{enumerate}

