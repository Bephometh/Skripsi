%versi 2 (8-10-2016) 
\chapter{Pendahuluan}
\label{chap:intro}
   
\section{Latar Belakang}
\label{sec:label}

IDE UNPAR adalah \textit{learning management system} berbasis web yang digunakan oleh UNPAR untuk membantu proses pembelajaran interaktif. IDE UNPAR bekerja dengan menyediakan mata kuliah yang diambil oleh mahasiswa secara virtual lengkap dengan peserta lain dari mata kuliah tersebut yang dapat mengaksesnya. IDE UNPAR juga membantu dosen merencanakan dan memantau proses pembelajaran. Mahasiswa juga dipermudah untuk melihat dan mengetahui proses dan tujuan pembelajaran dari suatu mata kuliah. 

Berdasarkan footer pada IDE UNPARR, IDE UNPAR dibuat dengan menggunakan \textit{Blackboard Open Learning Management System}\cite{IDEUNPAR} yang merupakan program berbasis Moodle, namun berdasarkan halaman bantuan \textit{Blackboard Open Learning Management System}, \textit{Blackboard Open Learning Management System} telah berganti menjadi \textit{Open LMS}\cite{Blackboard},sehingga penelitian ini akan befokus kepada Moodle. Moodle adalah \textit{learning management system} bersifat \textit{Open-source} yang dibuat menggunakan bahasa pemrograman \textit{PHP}. Moodle dilisensikan dibawah lisensi \textit{GNU GENERAL PUBLIC LICENSE Version 3, 29 June 2007}. Lisensi tersebut memperbolehkan adannya modifikasi terhadap program yang dilisensikan.

Moodle menyediakan \textit{source code} untuk \textit{learning management system} berbasis mobile. Moodle mobile memungkinkan penggunanya mengakses \textit{learning management system} berbasis Moodle web melalui perangkat mobile mereka. Pengguna Moodle mobile dapat mengakses \textit{learning management system} yang mereka gunakan dengan memasukkan \textit{URL} \textit{learning management system} dan memasukkan kredensial login mereka apabila diperlukan. Moodle mobile akan menampilkan data dan memberi akses yang serupa dengan apa yang ada pada \textit{learning management system} Moodle web. Moodle mobile dibangung dengan menggunakan \textit{Ionic Framework}. \textit{Ionic Framework} adalah sebuah \textit{Software development kit} untuk membuat aplikasi mobile dan desktop dengnan menggunakan teknologi seperti HTML, CSS dan \textit{Javascript}\cite{Ionic:intro}. Moodle mobile dilisensikan dibawah lisensi   \textit{APACHE LICENSE, VERSION 2.0}. Lisensi tersebut juga memperbolehkan dilakukannya modidfikasi terhadap \textit{source} dari aplikasi.



\section{Rumusan Masalah}
\label{sec:rumusan}
Rumusan masalah yang akan dibahas dalam penulisan skripsi ini adalah :
\begin{itemize}
	\item Bagaimana Moodle mobile IDE UNPAR dapat mengakses IDE UNPAR?
	\item Perbaikan apa saja yang dapat dilakukan untuk mempermudah penggunaan Moodle mobile?
	\item Bagaimana implementasi perbaikan tersebut ke dalam Moodle mobile?
\end{itemize}

\section{Tujuan}
\label{sec:tujuan}
Tujuan yang ingin dicapai dalam penulisan skripsi ini adalah :
\begin{enumerate}
	\item Menghubungkan Moodle mobile IDE UNPAR dengan Moodle web IDE UNPAR agar data yang ditampilkan sama.
	\item Melakukan \textit{hardcode} URL "https://ide.unpar.ac.id" agar saat aplikasi dibuka pengguna tidak perlu memasukkan alamat IDE UNPAR.
	\item Menganalisis lisensi dari Moodle dan apabila diperbolehkan merubah branding menjadi UNPAR.
	\item Menambahkan fitur baru pada aplikasi IDE UNPAR mobile.
	\item Meluncurkan aplikasi ke dalam Google Play dengan status \textit{Open testing}.
\end{enumerate}

\section{Batasan Masalah}
\label{sec:batasan}
Adanya masalah pada web IDE UNPAR yang menyebabkan aplikasi Moodle mobile tidak dapat mengaksesnya, maka diperlukan adanya batasan masalah yang jelas mengenai pembuatan aplikasi dan penulisan skripsi ini. Berikut merupakan batasan masalah untuk skripsi ini :
\begin{enumerate}
	\item Pengujian aplikasi dilakukan pada perangkat bergerak berupa telepon genggam tidak dilakukan pada \textit{desktop} atau \textit{tablet}.
	\item Pengujian hanya dilakukan pada platform Android.
\end{enumerate}

\section{Metodologi}
\label{sec:metlit}
Metode penelitian yang digunakan dalam skripsi ini adalah :
\begin{enumerate}
		\item Mempelajari Moodle mobile.
		\item Menganalisis lisensi dari Moodle mobile.
		\item Menyiapkan lingkungan pengembangan aplikasi.
		\item Menghubungkan aplikasi Moodle mobile dengan IDE UNPAR.
		\item Melakukan perubahan dan penambahan pada fitur Moodle mobile.
		\item Mengubah branding dari Moodle menjadi UNPAR.
		\item Menulis dokumen skripsi.
\end{enumerate}

\section{Sistematika Pembahasan}
\label{sec:sispem}
 Sistematika pembahasan pada skripsi ini adalah sebagai berikut :
 \begin{enumerate}
 	\item Bab 1 akan membahas latar belakang, rumusan masalah, tujuan, batasan masalah, meteodologi dan sistematika pembasahan.
 	\item Bab 2 akan membahas IDE UNPAR, Moodle dan Moodle Mobile Development
	\item Bab 3 akan membahas analisa dari Moodle, IDE UNPAR dan Moodle mobile
	\item Bab 4 akan membahas dimana saja perubaha pada proyek terjadi dan perancangan implementasi fitur-fitur dari umpan balik
	\item bab 5 akan membahas implementasi dan pengujian aplikasi pada perangkat bergerak.
	\item Bab 6 akan membahas kesimpulan dairi pengenmbangan aplikasi dan sara untuk pengembagnan seterusnya.
 \end{enumerate}