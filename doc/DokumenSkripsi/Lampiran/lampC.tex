%versi 3 (18-12-2016)
\chapter{Perubahan Pada File \texttt{src/config.json} untuk Menghubungkan ke Situs IDE UNPAR}
\label{lamp:C}

%terdapat 2 cara untuk memasukkan kode program
% 1. menggunakan perintah \lstinputlisting (kode program ditempatkan di folder yang sama dengan file ini)
% 2. menggunakan environment lstlisting (kode program dituliskan di dalam file ini)
% Perhatikan contoh yang diberikan!!
%
% untuk keduanya, ada parameter yang harus diisi:
% - language: bahasa dari kode program (pilihan: Java, C, C++, PHP, Matlab, C#, HTML, R, Python, SQL, dll)
% - caption: nama file dari kode program yang akan ditampilkan di dokumen akhir
%
% Perhatian: Abaikan warning tentang textasteriskcentered!!
%



\begin{lstlisting}[language=diff, frame=single, label ={auto-redirect}, caption = Perubahan pada file \texttt{src/config.json} ]

diff --git a/src/config.json b/src/config.json
index 919869fe652..daaf2aef5c5 100644
--- a/src/config.json
+++ b/src/config.json
@@ -68,10 +68,10 @@
         75.89,
         93.75
     ],
-    "demo_sites": {
-        "student": {
-            "url": "https:\/\/school.moodledemo.net",
-            "username": "student",
-            "password": "moodle"
-        },
-        "teacher": {
-            "url": "https:\/\/school.moodledemo.net",
-            "username": "teacher",
-            "password": "moodle"
-        }
-    },
+     "demo_sites" :[] ,
     "customurlscheme": "moodlemobile",
-    "siteurl": "https://moodledemo.pascal.id/",
+    "siteurl": "https://ide.unpar.ac.id/",
     "sitename": "",
     "multisitesdisplay": "",
     "sitefindersettings": {},
\end{lstlisting} 

