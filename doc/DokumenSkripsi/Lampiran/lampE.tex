%versi 3 (18-12-2016)
\chapter{Kode dan perubahan untuk fitur \textit{PDF Scanner}}
\label{lamp:E}

%terdapat 2 cara untuk memasukkan kode program
% 1. menggunakan perintah \lstinputlisting (kode program ditempatkan di folder yang sama dengan file ini)
% 2. menggunakan environment lstlisting (kode program dituliskan di dalam file ini)
% Perhatikan contoh yang diberikan!!
%
% untuk keduanya, ada parameter yang harus diisi:
% - language: bahasa dari kode program (pilihan: Java, C, C++, PHP, Matlab, C#, HTML, R, Python, SQL, dll)
% - caption: nama file dari kode program yang akan ditampilkan di dokumen akhir
%
% Perhatian: Abaikan warning tentang textasteriskcentered!!
%



\begin{lstlisting}[language=diff, frame=single, label ={fileuploader-helper}, caption = Perubahan pada \texttt{src/core/fileuploader/providers/helper.ts} ]
diff --git a/src/core/fileuploader/providers/helper.ts 
b/src/core/fileuploader/providers/helper.ts
index fe28bcb3e4..e69ee619aa 100644
--- a/src/core/fileuploader/providers/helper.ts
+++ b/src/core/fileuploader/providers/helper.ts
+    /**
+     * 
+     * @param maxSize Max size of the upload. -1 for no max size.
+     * @param upload True if file should be uploaded,
+     *  false to return to picked file.
+     * @param mimetypes List of supported mimetypes.  
+     * @return Promise solved when done.
+     */
+    scanImage(maxSize : number,
+    upload?:boolean, mimetypes?: string[]): Promise<any>{
+
+        const camOpts: CameraOptions = {
+            quality: 50,
+            destinationType: this.camera.DestinationType.FILE_URI,
+            correctOrientation: true
+        };
+
+            // Determine the mediaType based on the mimetypes.
+            if (imageSupported && !videoSupported) {
+                camOpts.mediaType = this.camera.MediaType.PICTURE;
+            } else if (!imageSupported && videoSupported) {
+                camOpts.mediaType = this.camera.MediaType.VIDEO;
+            } else if (CoreApp.instance.isIOS()) {
+                // Only get all media in iOS 
+	       //because in Android using this option 
+	      //allows uploading any kind of file.
+                camOpts.mediaType = this.camera.MediaType.ALLMEDIA;
+            }
+        }
+
+        return this.fileUploaderProvider
+	.getPicture(camOpts).then((path) => {
+            const error = this.fileUploaderProvider
+	    .isInvalidMimetype(mimetypes, path); 
+	     // Verify that the mimetype is supported.
+            if (error) {
+                return Promise.reject(error);
+            }
+                const html = '<html> <img src="'+ path +
+	        '" style="width: 100%; height 100%"> </html>';
+                this.logger.debug(html);
+ 	        var currentDate = new Date().getTime();
+                return cordova.plugins.pdf.fromData(html,{
+                    fileName : 'my-pdf'+currentDate,
+                    landscape : "portrait",
+                    type : "base64" 
+		//Using this type because the document 
+		//will be uploaded right away.
+                }).then((base64)=>{   
+                    //converting to blob
+                    const blob = this.base64ToBlob(base64);
+                    
+                    const contentType = 'application/pdf'
+                    const folderPath = "Download/my-pdf"
+ 		+currentDate+".pdf";
+                    return this.fileProvider.writeFile(folderPath, blob)
+		.then((fileEntry)=>{
+                        const options = this.fileUploaderProvider
+			.getFileUploadOptions(fileEntry.nativeURL, +	
+			'mypdf'+currentDate.getTime()+'.pdf',  
+			contentType, true);
+                        if(upload){;
+                            this.logger.debug("uploaded");
+                           return this.uploadFileEntry(fileEntry, true, 
+ 		        -1, upload, false, fileEntry.name);
+                        } else {
+                            // Copy or move the file 
+			to our temporary folder.
+                            this.logger.debug("Copy to temp");
+                            return this.copyToTmpFolder
+ 		         (fileEntry.nativeURL, 
+ 		         false, maxSize, 'pdf', options);
+                        }
+                    
+                    },
+                    (error) => {
+                        this.logger.error(error);
+                    });
+                },(error) => {
+                    const defaultError = 'core.fileuploader
+		.errorcapturingimage';
+                    console.error(error);
+                    return this.treatImageError(error, defaultError);
+                });
+        });
+    
+    }
+
\end{lstlisting}

\lstinputlisting[language=Java, caption=scanner-handler.ts, label={scanner-handler}]{./Lampiran/scanner-handler.ts} 

\begin{lstlisting}[language=diff, frame=single, label ={langindex.json}, caption = Perubahan pada file \texttt{langindex.json} ]
diff --git a/scripts/langindex.json b/scripts/langindex.json
index 0a3a21d16d2..2abd385d7a6 100644
--- a/scripts/langindex.json
+++ b/scripts/langindex.json
@@ -1600,6 +1600,7 @@
   "core.fileuploader.uploading": "local_moodlemobileapp",
   "core.fileuploader.uploadingperc": "local_moodlemobileapp",
   "core.fileuploader.video": "local_moodlemobileapp",
+  "core.fileuploader.scanner" : "local_moodlemobileapp",
   "core.filter": "moodle",
   "core.folder": "moodle",
   "core.forcepasswordchangenotice": "moodle",
\end{lstlisting} 

\begin{lstlisting}[language=diff, frame=single, label ={fileuploader-lang-eng}, caption = Perubahan pada file \texttt{src/core/fileuploader/lang/en.json} ]
diff --git a/src/core/fileuploader/lang/en.json
 b/src/core/fileuploader/lang/en.json
index 22d14df4a11..f0b31d17abf 100644
--- a/src/core/fileuploader/lang/en.json
+++ b/src/core/fileuploader/lang/en.json
@@ -25,5 +25,6 @@
     "uploadafile": "Upload a file",
     "uploading": "Uploading",
     "uploadingperc": "Uploading: {{$a}}%",
-    "video": "Video"
+    "video": "Video",
+    "scanner" : "Scan PDF"
 }
\end{lstlisting} 

